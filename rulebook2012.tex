%%%%%%%%%%%%%%%%%%%%%%%%%%%%%%%%%%%%%%%%%%%%%%%%%%%%%%%%%%%%%%%%%%%%%%%%%%%%%
%%% Ruebook RoboCup Logistics League Sponsored by Festo 
%%% Draft for 2013 Competition
%%%
%%% $URL$
%%% $Id$
%%% $Rev$
%%% $Author$
%%% $Date$
%%%
%%%%%%%%%%%%%%%%%%%%%%%%%%%%%%%%%%%%%%%%%%%%%%%%%%%%%%%%%%%%%%%%%%%%%%%%%%%%%

\documentclass[12pt,twoside]{article}

\usepackage[a4paper]{anysize}
\marginsize{2.5cm}{2cm}{2cm}{2cm}

\setlength{\marginparwidth}{1cm}
\usepackage{fourier}
%\newcommand\rulechange[1]{\begin{shaded}#1\end{shaded}\marginpar{\LARGE\danger}}

\newenvironment{rulechange}{%
  \def\FrameCommand{\fboxsep=\FrameSep \colorbox{shadecolor}}%
  \marginpar{\vspace*{2em}\LARGE\danger} \MakeFramed  {\FrameRestore}}%
 { \endMakeFramed}


\usepackage{framed,xcolor}
\colorlet{shadecolor}{gray!25}
\newcommand{\Robotino}{Robotino\textregistered}

%% GRAPHICSX %%%%%%%%%%%%%%%%%%%
\usepackage[pdftex]{graphicx}
\graphicspath{{figures/}}
\DeclareGraphicsExtensions{.pdf,.jpeg,.png,.JPG,.jpg}

%% TIKZ %%%%%%%%%%%%%%%%%%%
\usepackage{tikz}
\usetikzlibrary{arrows,shadows}
\usetikzlibrary{calc,positioning}
\usetikzlibrary{snakes,shapes}
\usetikzlibrary{shapes.callouts}

%% HYPERREF %%%%%%%%%%%%%%%%%%%
\usepackage{hyperref}

%% SVN-MULTI %%%%%%%%%%%%%%%%%%%
\usepackage{svn-multi}

%% TODONOTES %%%%%%%%%%%%%%%%%%%
\usepackage{todonotes}

%% TIMES  %%%%%%%%%%%%%%%%%%%
\usepackage{times}

%% TABLES %%%%%%%%%%%%%%%%%%%
\usepackage{tabularx}
\usepackage{multicol}
\usepackage{multirow}
\usepackage{calc}

%% INPUTENC %%%%%%%%%%%%%%%%%%%
\usepackage[utf8]{inputenc}
%%%%%%%%%%%%%%%%%%%%%%%%%%%%%%%%%%%%%%%%%%%%%%%%%%%%%%%%%%%%%%%%%%%%%%%%%%%%%
%%% Ruebook RoboCup Logistics League Sponsored by Festo 
%%% Draft for 2013 Competition
%%%
%%% $URL$
%%% $Id$
%%% $Rev$
%%% $Author$
%%% $Date$
%%%
%%%%%%%%%%%%%%%%%%%%%%%%%%%%%%%%%%%%%%%%%%%%%%%%%%%%%%%%%%%%%%%%%%%%%%%%%%%%%



\newcommand{\s}[1]{\ensuremath{S_{#1}}}
\newcommand{\p}[1]{\ensuremath{P_{#1}}}
\newcommand{\m}[1]{\ensuremath{M_{#1}}}
\newcommand{\T}[1]{\ensuremath{T_{#1}}}
\newcommand{\dg}[1]{\ensuremath{DG_{#1}}}
\newcommand{\TAG}[1]{\texttt{#1}}
\begin{document}


%%%%%%%%%%%%%%%%%%%%%%%%%%%%%%%%%%%%%%%%%%%%%%%%%%%%%%%%%%%%%%%%%%%%%%%%%%%%%
%%% Titlepage
\pagenumbering{roman}


\begin{titlepage}
  \vspace*{5cm}
  \begin{center}
    \begin{LARGE}
      2013 Draft of the Rulebook\\[2ex]
      for the\\[2ex]
      RoboCup Logistics League sponsored by Festo\\[4ex]
    \end{LARGE}
    \hrule
    
    \vspace*{4ex}
    \begin{Large}
      The Technical Committee\\[6ex]
    \end{Large}
  \end{center}
  \vspace*{4cm}
  
  \noindent
  Revision: \svn{$Rev$}\\[2ex]
  Author: \svn{$Author$}\\[2ex]
  Date: \svn{$Date$}\\[2ex]  
\end{titlepage}
\thispagestyle{empty}
\pagebreak
\cleardoublepage

%%%%%%%%%%%%%%%%%%%%%%%%%%%%%%%%%%%%%%%%%%%%%%%%%%%%%%%%%%%%%%%%%%%%%%%%%%%%%
%% Table of Contents
\setcounter{page}{1}
\tableofcontents
\newpage
\cleardoublepage


%%%%%%%%%%%%%%%%%%%%%%%%%%%%%%%%%%%%%%%%%%%%%%%%%%%%%%%%%%%%%%%%%%%%%%%%%%%%%
%% 
\setcounter{page}{1}
\pagenumbering{arabic}


\section*{Preamble}

This revised set of rules provides a precise definition of the
competition environment. The rulebook ensures the same and fair
circumstances for all participants, it however is not meant to dictate
or suggest the way of approach to fulfill the task. Furthermore it
concentrates on defining all necessary variables and necessities and
is not meant to visualize the competition itself. For general
information about the idea of this competition, please refer to the
information booklet supplied at the Festo Didactic homepage. Technical
information about the Robotino® platform and details to components
used in the construction of the competition area are combined in the
engineering reference and try the powerful RoboCup Simulator free of
charge, which are both supplied at the link above, too.

After a thriving season 2011 and the official introduction of this
league into the RoboCup portfolio, we look forward to new scale of
competition that will emerge from initiatives around the globe. In
2012 we will crown the first Logistics League World Champion. In order
to enable a competition at eye level among the teams, the focus of
this years’ iteration is to polish the 2011 Rulebook and to shift some
of the competition demands in order to achieve a continued improvement
of competition revenue. This being said, some of the future
developments have been tabled for this year to enable discussions
prior to or at the events lead by the teams themselves. It will remain
our goal to push for a competition suiting the industrial process.

Finally, no rulebook is perfect. Feel obliged to inform us about
issues you like to discuss or gaps that might have an impact on the
competition, so we can keep the necessity for rule discussions at the
RoboCup event to a minimum. We are open for all kinds of suggestions;
the set of rules will be fixed at 01/01/2012 and revised in April 2012
once the German Open is over. We also encourage all interested teams
to apply as a member of the Organization Committee (OC) for both the
German Open and the RoboCup 2012 in Mexico.


\todo[inline]{Fill in contact details about the TC and OC}

%%%%%%%%%%%%%%%%%%%%%%%%%%%%%%%%%%%%%%%%%%%%%%%%%%%%%%%%%%%%%%%%%%%%%%%%%%%%%
\section{Agreements \& Regulations}

The Logistics League follows a certain design philosophy. All teams
are obliged to use the Robotino® robotic system of Festo Didactic Gmbh
\& Co. KG with certain freedoms and limitations. The usage of both
revisions, namely 2009 and 2010 are in order, see chapter 4 for
details.


%%%%%%%%%%%%%%%%%%%%%%%%%%%%%%%%%%%%%%%%%%%%%%%%%%%%%%%%%%%%%%%%%%%%%%%%%%%%%
\section{The Task –“A challenge of precision within a flexible
  deployment system”}

Our aim is to simulate autonomous guided vehicles (AGV). In opposition
to regular automatic guided vehicles, a team, consisting of a maximum
of three Robotinos®, shall complete the following task without a
control station or human interference as successful as possible
competing with a second team against the clock.

The main task is a 3-staged production cycle with self-crafted
intermediate products and the transport of the final product to the
designated zone. This is the genuine goal and will be rewarded
considerably higher than partial fulfillment of the task. Although the
distribution and utilization process of the different machine types is
known to the teams, the reference which production machine is of which
machine type is not. Therefore, a part of the task is to discover the
machine types of as many machines as suitable. In order to complete
the production cycle, it is required to produce the three
subassemblies step-by-step. The factory’s capacity theoretically
allows production of resources for two products at the same time. The
whole factory area can be used as an intermediate storage. Finally,
the successfully assembled product has to be delivered to the correct
delivery gate and get dismounted into its delivery slot. The factory
area has to be treated in the best possible way. Theoretical damage
will result in minor punishment. This includes machines as well as
pallet carriers.

\subsection{Production portfolio}
\begin{table}[h]
  \centering
  \begin{tabular}{c|c|c|c}
    \multicolumn{1}{c}{Subassembly} & \multicolumn{1}{c}{Deployable} & 	\multicolumn{1}{c}{Prerequisites} & \multicolumn{1}{c}{Result}\\\hline
    \s{0} &	\m{1}, \m{2}, \m{3}, \dg{} & 	none &	\s{1} or \TAG{consumed}\\
    \s{1} &	\m{2}, \m{3}, \dg{} &  	\s{0} & 	\s{2} or \TAG{consumed}\\
    \s{2} & \m{3}, \dg & 	\s{0}, \s{1} &	\p{}\\ %
    \TAG{Express Good} &  	\m{1} & none &  \TAG{Finished Express Good}\\
    \hline
  \end{tabular} 
  \caption{The production table}
  \label{tab:production-table}
\end{table}






\tikzstyle{M}=[rectangle, draw=blue, thick, fill=blue!20, text width=2em,align=center, rounded corners, 
  minimum height=2em]
%
\tikzstyle{P}=[draw=red, thick, circle,fill=red!20, minimum height=1.5em]

\begin{tikzpicture}[level distance=4em,sibling distance=5em]
  \node[P] {$S_1$}  [grow=left,thick]
  child { node[M] {$M_1$}
    child { node[P] {$S_0$}}};
\end{tikzpicture}
\hspace*{1em}
\begin{tikzpicture}[level distance=4em,sibling distance=5em]
  \node[P] {$C$}  [grow=left,thick]
  child { node[M] {$M_1$}
    child { node[P] {$S_0$}}};
\end{tikzpicture}

\begin{tikzpicture}[level distance=4em,sibling distance=5em]
  \node[P] {$C$}  [grow=left,thick]
  child { node[M] {$M_2$}
    child { node[P] {$S_0$}}
    child { node[P] {$S_1$}}
    child { node[P] {$S_2$}}
    };
\end{tikzpicture}

\begin{tikzpicture}[node distance=4em,level distance=4em,sibling distance=5em]

  \node[P] (p) {$P$}  [grow=left,thick]
  child { node[M] (m) {$M_3$}
    child { node[P] {$S_0$}}
    child { node[P] {$S_1$}}
    child { node[P] {$S_2$}}
   };
  
\end{tikzpicture}




% \newcommand{\varone} {
%   \begin{tikzpicture}[level distance=4em,sibling distance=5em]
%     \node[P] {$P$}  [grow=left,thick]
%     child {node[M] {$M_3$} 
%       child {node[P] {$S_0$}} 
%       child {node[P] {$S_1$}
%         child {node[M] {$M_1$} 
%           child {node[P] {$S_0$}}
%         }
%       } 
%       child {node[P, yshift=-1em] {$S_2$}
%         child {node[M] {$M_2$} 
%           child {node[P] {$S_0$}}
%           child {node[P] {$S_1$}
%             child {node[M] {$M_1$}
%               child {node[P] {$S_0$}}
%             }
%           }
%         }
%       }
%     };
%   \end{tikzpicture}
% }

% \newcommand{\varthree} {
%   \begin{tikzpicture}[level distance=4em,sibling distance=5em]
%     \node[P] {$P_2$}  [grow=left,thick]
%     child {node[M] {$M_3$} 
%       child {node[P] {$S_0$}}
%       child {node[P] {$S_2$}}
%       child {node[P] {$S_2$}}      
%     };
%     \end{tikzpicture}
% }

% \newcommand{\vartwo} {
%   \begin{tikzpicture}[level distance=4em,sibling distance=4em]
%     \node[P] {$P$}  [grow=left,thick]
%     child {node[M] {$M_3$} 
%       child {node[P] {$S_0$}}
%       child {node[P] {$S_1$}}
%       child {node[P,label={[label distance=-6pt,fill=red!20]270:{\footnotesize $0.5$}}] {$S_2$}}
%       child {node[P,label={[label distance=-6pt,fill=red!20]270:{\footnotesize $0.5$}}] {$S_2$}}      
%     };
%     \end{tikzpicture}
% }


% \begin{tikzpicture}
%   \matrix[anchor=east]{
%     \node (p1) {\varone};\\
%     \node (p2) {\vartwo};\\[2ex]
%     \node (p3) {\varthree};\\
%   };
% \end{tikzpicture}




\subsubsection{The production table}

The table above shows the production table concerning the main
challenge; the three staged production process as well as the express
good challenge. The main challenge can be repeated as long as enough
pallet carriers can be provided to complete the cycle. The different
machine types are specified in Sect.~\ref{sec:machines}.


\subsubsection{Express good}

The express good challenge will start as soon as a pallet carrier has
been inserted into the express good slot by the referee. The challenge
requires a fast paced processing and delivery within the time
requested. The challenge also requires prior knowledge concerning the
machine distribution.

\section{Competition Area}

The point of origin for each statement within this rulebook that uses
relative coordinates is the bottom left corner of the competition area
namely the corner near the recycling unit to the left of the outgoing
goods area. All indicated sizes of mark-ups are to be considered
outside dimensions.

The competition area is a 5.625m * 5.625m large arena with several
RFID-mounted machines, mark-ups, a stock of raw-material and a
delivery zone. It is surrounded by boards, 0.5 m of height to reduce
object interference from outside the area. The default width for
mark-ups is 19 mm, the default color is black.

The factory area spans across 5.625 m x 4.8 m. There are two
boundaries, set by two mark-ups, 0.4 m from the top and bottom borders
of the competition area. The factory area is joined by two opposing
0.4 m x 1.0 m zones at the top and bottom middle. The top zone is
painted “blue” marking the input store area with the express good
insertion point to the right. The insertion point is 0.6 m of width
with the insertion slot in its middle. This spot is a 0.1 m times 0.1
m empty square that will be equipped with a pallet carrier to start
the express good challenge. The space between the recycling unit and
the express good insertion point, as well as the space to the left of
the input store area is called robot insertion area.

The bottom zone is marked “green” and houses the three delivery gates.
Each gate is of 0.3m width with 0.1m space to the next delivery gate
separated by black mark-ups.

The delivery gates feature one signal per gate placed middle of each
gate zone. The delivery slot resides below a unit that is identical in
construction to a production machine. The three RFID devices within
these gates feature a black centered square of 0.1 m x 0.1 m called
delivery slot and residing exactly below the RFID device. Only a
pallet carrier that is delivered into the slot completely will be
considered for scoring, i.e. no part of the pallet carrier may leave
the outer slot boundary.

A total of 13 machines are placed within the competition area. 10
machines representing the 3 staged production processes, 2 machines to
recycle consumed pallet carrier and 1 test station. The machines of
the production process are placed within the factory area as stated in
the table below. They are aligned in a 90$^\circ$ angle. Each production
machine resides in the centre of a squared machine space spanned by
standard mark-ups with 0.6 m each side.

The 3 additional machines are arranged in the corners of the
competition area, and aligned 45$^\circ$, facing the center of the factory
area. The top left corner will remain empty.

\begin{figure}[h]
  \centering
  \includegraphics[width=0.8\linewidth]{601px-FLC2012_p}
  \caption{Competition Area}
  \label{fig:competition-area}
\end{figure}



\subsection{Coordinates of the Production Machines}
\label{sec:coordinates}

\begin{table}[h]
  \centering
  \begin{tabular}{l|r|r}
    \multicolumn{1}{l}{Number} & \multicolumn{1}{l}{$X~[m]$} & \multicolumn{1}{l}{$Y~[m]$}\\ \hline
    Machine 1 & 1.68 & 1.68 \\
    Machine 2 & 3.92 & 1.68 \\
    Machine 3 & 0.56 & 2.80 \\
    Machine 4 & 1.68 & 2.80 \\
    Machine 5 & 2.80 & 2.80 \\
    Machine 6 & 3.92 & 2.80 \\
    Machine 7 & 5.04 & 2.80 \\
    Machine 8 & 1.68 & 3.92 \\
    Machine 9 & 3.92 & 3.92 \\
    Machine 10 & 5.04 & 3.92 \\
    Recycling unit 1 & 0.20 & 0.20 \\
    Recycling unit 2 & 5.40 & 5.40 \\
    Test station &	5.40  &0.20 \\
    Express good insertion point / slot & 	3.60 & 	5.35 \\
    Delivery slot 1 & 	3.15 & 0.26 \\
    Delivery slot 2 &	2.80 & 0.26 \\ 
    Delivery slot 3 &	2.45 & 0.26 \\\hline
  \end{tabular}

  \caption{Coordinates of production machines}
  \label{tab:coordinates}
\end{table}

\subsection{The Pallet Carrier – Puck}
The data carrying RFID tag is mounted to a hockey puck. The tournament
puck features a diameter of 7.5cm.

\begin{figure}[h!]
  \centering
  \includegraphics[height=3cm]{125px-Puck}
  \caption{Puck}
  \label{fig:puck}
\end{figure}

\subsection{Machines}
\label{sec:machines}

\subsubsection{General Information}

\begin{figure}[h]
  \centering
  \includegraphics[height=6cm]{Machine}
  \caption{Machine}
  \label{fig:machine}
\end{figure}

All machines are identical devices consisting of 
\begin{itemize}
\item one plate housing the RFID read/write device and
\item one signal unit according to the figure above.
\end{itemize}

They share the same design and the same RFID device, with the overall
size of 280 mm height, 160 mm of width and 100 mm of depth, see the
engineering reference for further details. The default operating mode
of all machines implies that only the green LED is turned on. This
signals the machine being ready for input. The reading and writing
process generally is a delicate process. To avoid corruption of the
data carrier, it should not leave the working range of the RFID device
once the processing or consuming is started. To enable the production
process it is necessary to transport the pallet carrier accurately to
the RFID device. A consumed pallet carrier has to stay within the
machine space borders(no part of the puck being outside the mark-up)
until the production cycle of that very machine has been
completed. Production resulting from violating this requirement is
considered junk and will not be rewarded. The machine always processes
the required pallet carrier delivered last, all prior components will
be consumed. All machines will start processing the data carrier as
soon as they enter the diameter named below and change their operating
mode according to the tables provided.


\subsubsection{Production Machines}
After processing the current data carrier: Yellow LED turned on The
machine has finished processing the current data carrier and is
waiting for the next subassembly. Green LED turned on The machine has
finished the work order and is ready to receive the next batch of
carriers. In order to complete the machines’ work order the input
materials have to be delivered one-by-one into the RFID device’s
action range. Multiple data carriers in range of the device will
result in erroneous behavior of the device. Consumption of materials,
like \s0 used in the production of \s2, will take 2 seconds. Unloading
the machine can be done immediately after the operating mode changes
away from processing. As long as the machines are used properly, they
will not produce any junk.

\begin{table}[h]
  \centering
  \begin{tabularx}{\linewidth}{p{0.35\linewidth}|X}
    \multicolumn{1}{l}{Optical Feedback} &
    \multicolumn{1}{l}{Operating mode} \\ \hline
    All LEDs turned off &  	The machine is physically offline, caused by a real error which should not happen during the competition. \\
    Red LED turned on &  	The machine is out of order \\
    Green LED turned on &  	The machine is idle and ready.\\
    Green and yellow LED turned on &  	The machine is processing or consuming the current data carrier. \\
    Yellow LED flashing(at 2 Hz) & The machine detects wrong material.
    This can be caused by data carriers that are already consumed,
    subassemblies that do not fit to this machine type’s work order or
    corrupted data carriers. \\\hline
  \end{tabularx}

\bigskip
  \begin{tabularx}{\linewidth}{l|X|X|X|l}
    \multicolumn{1}{l}{ Type} & \multicolumn{1}{l}{Distribution} & \multicolumn{1}{l}{Input} & \multicolumn{1}{l}{Output} & \multicolumn{1}{l}{(Final)processing time[s]}\\\hline
    \m1 & 4 times & \s0 (Raw-material) & \s1 & $WT_1 = 3 \mbox{ to } 8~\mathrm{sec}$\\
    \m2 & 3 times & \s0; \s1 \s2; & one consumed container & $WT_2 = 15 \mbox{ to } 25~\mathrm{sec}$\\
    \m3 & 3 times &	\s0;\s1;\s2 & Product; two consumed containers & $WT_3 = 40 \mbox{ to } 60~\mathrm{sec}$
  \end{tabularx}

  \caption{Production Machines}
  \label{tab:production-machines}
\end{table}



The distribution describes how many machines of the respective types
will be randomly placed resulting in a the total of 10 court machines.



\subsubsection{Recycling Unit (former market place)}
The recycling unit processes all supplied loading carriers back to
raw-material (\s0) within 2 seconds.

\begin{table}[h]
  \centering
  \begin{tabularx}{\linewidth}{p{0.35\linewidth}|X}
    \multicolumn{1}{l}{Optical Feedback} &\multicolumn{1}{l}{Operating
      mode}\\\hline
    All LEDs turned off & 	The machine is physically offline, caused by a real error which should not happen during the competition.\\
    Red LED turned on & 	The machine is out of order\\
    Green LED turned on & 	The machine is idle and ready.\\
    Green and yellow LED turned on & The machine is processing the
    current data carrier.\\\hline
  \end{tabularx}
  \caption{Recycling Unit}
  \label{tab:recycling-unit}
\end{table}



\subsubsection{Reading device}
The reading and visualizing the data carriers content happens almost
instantly after delivering the pallet carrier to the action range of
the device.

\begin{table}[h]
  \centering
  \begin{tabularx}{\linewidth}{p{0.35\linewidth}|X}
    \multicolumn{1}{l}{Optical Feedback} &\multicolumn{1}{l}{Stored
      data on the data carrier}\\\hline
    Green LED turned on &	The station is ready to read the next data carrier.\\
    All LEDs turned off &	Consumed pallet carrier.\\
    Yellow LED turned on &	Raw-material (\s0) \\
    Red and yellow LEDs turned on & 	Subassembly 1 (\s1)\\
    Red LED turned on &	Subassembly 2 (\s2)\\
    All LEDs turned on & The final product (\p{})\\\hline
  \end{tabularx}
  \caption{Reading Device}
  \label{tab:reading-device}
\end{table}

\subsubsection{Delivery Gates} \begin{table}[h]
  \centering
  \begin{tabularx}{\linewidth}{p{0.35\linewidth}|X}
    \multicolumn{1}{l}{Optical Feedback} &\multicolumn{1}{l}{Stored
      data on the data carrier}\\\hline
    Red LED turned on & This delivery gate is inactive.\\
    Red and green LED turned on & This gate is active, namely the designated gate.\\
    \hline
  \end{tabularx}
  \caption{Delivery Gates}
  \label{tab:delivery-gates}
\end{table}

As soon as a pallet carrier is successfully delivered to the active
gate, it will show the state of the data carrier as described above.
This state will only long for some seconds and only for scoring
reasons. There will be only one active gate at a time.


%%%%%%%%%%%%%%%%%%%%%%%%%%%%%%%%%%%%%%%%%%%%%%%%%%%%%%%%%%%%%%%%%%%%%%%%%%%%%
%%%

\section{The Robotino® System}

All participants have to design their competition Robotinos® within
the following specifications:

Any kind of sensors can be changed or added to the Robotino® platform.
However, it is not possible to implement sensors that require
modifications outside the Robotino® area (e.g. Northstar, indoor GPS).
It is furthermore strictly forbidden to implement any kind of RFID
device into the Robotino®. There must be no changes to the controller
or mechanical system. The pushing device is defined as a passive,
non-mechanical load handling attachment. The robots peripherals must
neither exceed the maximum total height of 0.7 m nor the 0.4 m
diameter of the body cylinder. The only exception to this is the one
default mounted pushing device per robot. The pushing device can be
modified; it however must not exceed the following outside dimensions:
0.25 m x 0.15 m x 0.05 m.

\begin{rulechange}
It is allowed to install additional computing power on the
Robotino. This may either be in form of a notebook/laptop device or
any other copmuting device that fits into the size requirement of the
\Robotino{} system. Furthermore, it is allowed to communicate with an
additional computing device off-field. This device may be used for
team coordination etc.
\end{rulechange}

For a detailed technical description, refer to the Engineering
Specifications chapter 1.1 

\subsection{Communication}

Each robot has to operate autonomously. The communication between the
robot and the device responsible for the Start/Stop command, as well
as all communication amongst the robots has to be realized using the
Wi-Fi connection. The program controlling the robot has to be executed
locally by the robot itself. It is strictly forbidden to use any kind
of external server acting as command point. The robots are allowed to
share information with other devices, but must receive nothing else
but the “start”, “pause” and “stop” command from units other than the
2 fellow robots. This specifically excludes: Usage of processed image
data created outside of the robots A central communication that
requires a device other than the three Robotinos® A permanently
established connection between the command device and the Robotinos®.

Please refer to Sect.~\ref{sec:radio-interference} for further details.

%%%%%%%%%%%%%%%%%%%%%%%%%%%%%%%%%%%%%%%%%%%%%%%%%%%%%%%%%%%%%%%%%%%%%%%%%%%%%
%%%

\section{Tournament}
\subsection{Setup}

A match is defined by two contesting teams competing at two separated
identical competition areas. Each match lasts 15 minutes with 5
minutes of setup time unless stated otherwise by the organization
team. All settings, including the EGC and random events, will be the
exact same for both parties of a match.

\subsection{Team setup}
\label{sec:team-setup}
No team member is allowed to enter the competition area prior to or
during a match. The robots can be set up within the robot insertion
area as long as they are outside the factory area and have not been
elevated into their autonomous state. During a match the manipulation
is limited to adjustments on sensors, checking cable connections and
the boot or shut down procedure. A team can ask the referee to shut
down the robot. If this motion has been forwarded within the first 15
seconds of the very robots movement and without this robot scoring
points, the referee will move it to a point of insertion of the team’s
choice, once. Otherwise or on second occasion the robot will be
removed from the competition area. Resetting or removing a robot will
not cause an interruption of the game. The referee will only interrupt
the game if there is no other way to reset the robot without
interfering with the other ongoing processes. Once removed from the
competition area the robot cannot be reinserted during the same match.
A team can also decide to remove their robot from the competition area
at any time of the match.

\subsection{Setup environment}
\subsubsection{Machine Initialization}

The physical distribution of the production machines is fixed. Their
alignment will be randomized during the event setup but will stay that
way through the whole event. The machine type of each production
machine will be randomized prior to each match. The processing time of
each machine type will be determined in the same way, so the waiting
time during a match will be static for each machine of the three
machine types (e.g. all M1 could have 7 seconds processing time). The
active delivery gate will also be randomized prior to each match but
during a match the active gate can switch.


\subsubsection{Radio Interference}
\label{sec:radio-interference}
The referee will induce a connection breakdown between the command
unit and the Robotinos® at certain points of a match. This will not
affect the Wi-Fi connection between the Robotinos® and will neither
happen during the first minute of a match. Once switched off, the link
between LAN and Wi-Fi will stay severed for 100 seconds. The link will
be reactivated swiftly in case of emergency to interrupt the autonomy
of the process.


\subsection{Match startup}

All matches will start at the exact time scheduled by the organization
team. From this point on, the teams involved are allowed to start
their robots to work autonomously. This can be done by one click per
robot on any kind of interface.

\subsection{During a match}

The referee can interrupt the match at any time. Then, both teams have
5 seconds to stop all robot movement. The match time will be paused
during the interruption. A team can decide to stop the autonomy
process of each robot individually at any time of the match. Doing so
has to be announced notably in order to inform the referee, as this is
considered a shut down request according to
\ref{sec:team-setup}. Robots that do not stop within the time limit
will be treated in the same way.

\subsubsection{Out-of-order}

The downtime generator will take down a maximum of two machines out of
the pool containing production machines and recycling units. It will
do so at random points of time. There will be 6 to 8 of such triggered
events during a match. The machines affected will remain out of order
for 60 to 120 seconds. Every machine can only be forced out of order
once per match. If the machine turns offline during processing or
consumption of mounted a pallet carrier, it will afterwards resume the
process.

\subsubsection{Express good challenge}

The challenge will be induced by the referee seeding one loading
equipment into the express goods slot. The challenge then has to be
completed within 120 seconds. The loading equipment has to be
processed at a machine of type 1 with the corresponding WT1 as
processing time. Prior to handling the express good, the designated
machine has to be put into an identified state. This is done by a raw
material that was processed in this very machine earlier during the
on-going match. The identification process does not have an effect on
the further usage of the produced subassembly 1, it can be used
naturally. After processing the express good has to be delivered to
the active delivery gate.


\subsection{Mode}
\subsubsection{Tournament specifications}

The tournament features two stages with the first stage being done in
league form with several sequels orientating at the number of
participants and a second stage with playoffs featuring the top 4
teams.

Each match will be resulted with the score of each team. The winning
team will be awarded 2 major points. In case of a draw both teams will
be awarded with 1 major point. In case both teams scored zero points,
no major points will be awarded.

In case of a draw within the playoffs, the game time will be extended
by 5 minutes unless both teams scored zero points.

If this extension leads to a draw too the overall regular points of
the teams will determine the match winner. If the overall points are
equal too, a direct comparison between the teams in question will
decide. If this fails to resolve the situation, the teams will
approach a coin toss to determine the winner.

The detailed seeding will be created at the event. Although the idea
is to allow each participant to challenge each other team the league
can be adjusted to meet time requirements.


\subsubsection{Tournament challenge}
Proceeding to the playoffs will result in the following changes to the EGC setup:
% \begin{table}[h]
%   \centering
%   \begin{tabularx}{\linewidth}{l|X}
%     \multicolumn{1}{l}{Phase} &\multicolumn{1}{l}{Remark}\\\hline
%     \multirow{7}{*}{League phase} & 
%     \begin{itemize}\itemsep-3pt
%     \item The active delivery gate will not switch during a match.
%       % 
%     \item There will be 3 express good challenges.
%       % 
%     \item     No challenge will be initialised during the first or within the last two match minutes.
%     \end{itemize} \\
%     \multirow{9}{*}{Playoffs} &
%     \begin{itemize}\itemsep-3pt
%     \item The active delivery gate will swap twice - after minutes 7
%       and 11. A delivery made to the old active gate will be still
%       valid for the next 10 seconds.
%       % 
%     \item There will be 4 express good challenges. 
%       % 
%     \item No challenge will be initialised during the first or within
%       the last two match minutes.
%     \end{itemize}\\\hline
%   \end{tabularx} 
%   \caption{Tournament Phases}
%   \label{tab:phases}
% \end{table}

\paragraph{League phase}~\\
\begin{itemize}
\item The active delivery gate will not switch during a match.
\item There will be 3 express good challenges.
\item No challenge will be initialised during the first or within the
  last two match minutes.
\end{itemize} 

\paragraph{Playoffs}~\\
\begin{itemize}
\item The active delivery gate will swap twice - after minutes 7
  and 11. A delivery made to the old active gate will be still
  valid for the next 10 seconds.
\item There will be 4 express good challenges. 
\item No challenge will be initialised during the first or within
  the last two match minutes.
\end{itemize}
     

In both cases, delivered or timed out pallet carriers will be removed
from the game by the administration and therefore cannot be recycled.

\subsubsection{Wi-Fi regulations}

In order to provide the optimal possible solution for wireless
communication during the event, all teams are required to use the 5
GHz Wi-Fi equipment. They are furthermore required to connect their
Robotinos® Wi-Fi unit to the access point provided. All teams can also
relay on Wi-Fi clients supplied by Festo but are not required to. A
detailed description concerning the infrastructure can be found in
chapter 1.8 of the Engineering Specifications.


\subsubsection{Task fulfilment?}

The following table provides the itemized clearance of all task
related processes.
\begin{table}[h]
  \centering
\begin{tabularx}{\linewidth}{p{6em}|X|p{4em}}
  \multicolumn{1}{l}{Subtask } &\multicolumn{1}{l}{Main challenge} &
  \multicolumn{1}{l}{Scoring [Point]}\\\hline
  Produce \s2 & Finish the work order of a machine type 2 & $+4$\\
  Produce \p  & Finish the work order of a machine type 3 & $+12$ \\
  Deliver & Deliver the final product to the designated loading zone & $+5$\\
  Recycle & Clean up a polluted machine (\m2 or \m3) by recycling all
  of the 3 consumed loading carriers. Partial recycling(\m3) will not
  be rewarded.&
  \multirow{3}{*}{$\begin{array}{l}+3 - \m2\\
      +6 - \m3\end{array}$}\\
  Sum &Total points a team will receive for a produced and correctly
  delivered final product with its consumed loading carrier
  recycled. & $30$\\\hline
  \end{tabularx}  

\bigskip
\begin{tabularx}{\linewidth}{p{6em}|X|p{4em}}
    \multicolumn{1}{l}{Subtask } &\multicolumn{1}{l}{EGC - Expressgood Challenge} &
    \multicolumn{1}{l}{Scoring [Point]}\\\hline
    Finish the EG &	Deliver the EG to a machine of type 1 and process the express good in time if the machine type was identified by an earlier production process. &	+5\\
    Deliver the EG & Deliver the processed express good to the active
    delivery gate in time. &  +10\\
    Sum &  In time delivery of a correct
    express good. & 15\\\hline
  \end{tabularx}  

  \caption{Scoring Scheme}
\end{table}



\subsection{Penalties}

This catalogue represents the decision basis of the referee without
being exhaustive or binding.

\begin{table}[h!]
  \centering
  \begin{tabularx}{\linewidth}{l|X}
  \multicolumn{1}{l}{Issue} &\multicolumn{1}{l}{Sanction}\\\hline
  Premature movement & No robot is allowed to move until the referee
  announced the start of the match The faulty robot will be grounded
  for
  2 minutes\\[1ex]
%
  Damaging factory equipment & Theoretical damage to the “real”
  factory equipment as a result of collisions and negligent actions.
  The team will be punished with a score reduction. The total score
  cannot  drop below zero.\\[1ex]
%
  Not showing up & A team not showing up at all. The team will be
  removed from the tournament unless the team leader can provide a
  sincere explanation\\[1ex]
%
  Breaking a minor rule & A rule infringement with none or little
  impact on the team performance The team will receive a warning or a
  small  score reduction\\[1ex]
%
  Breaking a major rule & A rule infringement with considerable impact
  on the team performance or competition mechanics. The referee will
  decide upon calling a team vote or imposing an adequate punishment.\\[1ex]
%
  Arguing with the referee & There will be no discussions during a
  match. Each team can make a motion to protest a certain match and
  its result which will be dealt with after the match. There will be a
  warning. Continued disregard will result in a time punishment to the
  team’s current or next match.\\[1ex]
%
  Disregarding rules of conduct & Following the rules of conduct
  should be self-explanatory Upon disregard, the referee will impose
  sanctions ranged from time punishments to the team’s complete
  removal from the tournament.\\\hline
  \end{tabularx}  
  \caption{Infringements}
  \label{tab:infringements}
\end{table}



\begin{rulechange}
  \subsection{Technical Challenge}
  
  Within the league, the technical advances should be documented from
  year to year. Therefore, the Technical Challenge is introduced.
  Each participating team should prepare for participating in any
  number the following tasks:


  \paragraph{Collision avoidance.~}
  The robot is to show that it avoids other obstacles and robots.
  Therefore the robot must drive from input storage to the delivery zone.
  However, the paths between the input storage and the delivery zone will
  be blocked randomly by static obstacles and the robot must not touch any
  of the obstacles. For touching an obstacle, the team receives a penalty.
  The fastest team with the fewest penalty points to reach the delivery
  zone wins this challenge. All other teams are ranked according to
  how fast they were and how many penalties they conceived.
  
  \paragraph{Whac-a-Mole.~}
  A single robot is placed somewhere on the field. It has to detect
  the single shining signal unit on the play field. As soon as it puts
  a puck underneath it, the signal unit is switched off and another
  random signal unit is switched on. The goal of the challenge is to
  switch as many signal units as possible within a given time
  frame. Teams are ranked according to the number of turned-off
  signals.
  
  \paragraph{Free challenge.~}
  Each team will be given 5 minutes to showcase their robot team, e.g.
  show some new robotics developments. The team leader of
  non-presenting team will judge the performance and rate it with
  points between 0--10.  The team with the most points will win this
  challenge. The other teams are ranked in decreasing point order.


  The technical challenge is conducted in the following way: The
  teamleader of each participating team agree on a date and time
  during the tournament for the Technical Challenge in their first
  Teamleader Meeting. Each team can register for any of the
  challenges. All teamleaders have to be present at the time of the
  challenge to judge the other teams. The OC is responsible to conduct
  the Technical Challenge and can appoint teamleaders to support in
  conducting the challenges. The best team will be awarded at the
  closing ceremony.

\end{rulechange}

%%%%%%%%%%%%%%%%%%%%%%%%%%%%%%%%%%%%%%%%%%%%%%%%%%%%%%%%%%%%%%%%%%%%%%%%%%%%%
%%%

\section{Development / Vision}

This section is meant to enable discussions and support investment
decisions for future soft- and hardware acquisitions.

\subsection{Short term ideas}

These are ideas that could still be incorporated into the rulebook of
Istanbul 2011. 

\subsubsection{Scripted, dynamic obstacles}

On the way to fully dynamic obstacles this iteration implies a fully scripted administration controlled Robotino® that follows implicit movement rules that are known to all participants.

\subsection{Midterm planning}

Additions and alterations for future iterations of this competition


\subsubsection{Various Production programs}

A part from the three-staged production process, various goods with
different work orders and specifications (e.g. top speed, delivery
strategies...) could be part of the challenge. This addition seems to
be heavily dependent on Sect.~\ref{sec:supp-flow}.

\subsubsection{Various order strategies}

A delivery could consist of more than one final product, it could be required to deliver a batch of products, maybe within a certain time span, to complete the loading and receive extra points. Also, the different delivery gates could obtain a predefined shipping list, for example gate 1 requiring 2 Products, 2 M2 and 1 M1, maybe in the correct order to enable FIFO, LIFO or other delivery strategies.

\subsubsection{Simulation league}

Since there is only the annually world championship and maybe a
regional preregistration, a simulation platform could be provided,
where the software framework of teams can be used to compete with
other teams. Additionally a branch of simulation could be created that
focuses on the simulation of many AGV and a huge production area in
order to compete on scalability.

\subsubsection{Introducing a supportive flow of information}
\label{sec:supp-flow}

As the current task only deals with the material stream, it is heavily limited to a simple static task. In order to enable a flow of information that transports complex orders, a combined effort should focus on implementing a data interface that can be used by all teams today and in the future. As this would be a giant leap towards the industrial application, a general discussion and a lot of effort has to be invested into this issue.


\subsection{Long term Vision}

Ideas, dreams and ideology that inspire the future development.

\subsubsection{Complex production machines}

As there are more ways to interact with a machine than mounting and dismounting a loading carrier, it is possible to develop new machine types that look different and are completely different to handle.

\subsubsection{Collaborative Production}

Teams could be required to cooperate with another team to enable a
combined supply chain. 

\subsubsection{Opponent controlled dynamic obstacles}

No scripted obstacle can truly represent challenges of the industrial
application. In the long run, an opposing team has to be reinserted
that is allowed and requested to anticipate the logistic processes in
real time in order to create worst case scenarios for the teams.

\subsubsection{Interfacing ERP / SCM}

The interface used to present orders could be back ended with software
from real business applications like ERP, PPS, WHM and SCM.

\subsubsection{JIS / JIT implementation}

With complex production processes and several other achievements and
upgrades it could be useful to implement JIS and JIT tasks and
procedures into the LL, requiring delivery strategies like LIFO, FIFO
and certain time windows.



 
\end{document}
