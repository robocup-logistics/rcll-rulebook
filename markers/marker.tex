\documentclass{article}
\usepackage{geometry}
\geometry{a4paper}
\usepackage{tikz}
\usetikzlibrary{positioning,calc}
%creates tags in landscape format
\begin{document}
\begin{tikzpicture}[x=1cm,y=1cm,overlay, remember picture,
                    shift=(current page.south west)]
  %13/13.5 (old tag width was 13.5 cm, new one is 13
  \def\scaletag{0.96296296296}
  \node[rotate=90,font=\Huge] at (10,5) {\textbf{ @ID@}};
  \coordinate (anchor) at (1,1);
  \coordinate (lb) at (0,0);
  \coordinate (tr) at (21,29.7);
  \useasboundingbox (lb) rectangle (tr);
  \coordinate[above=277mm of  anchor] (h1);
  \coordinate[right=22mm of  h1] (h2);
  \coordinate[above right=215mm and 163mm of anchor] (h3);
  \coordinate (tag-left-bottom) at (4.5,8.7);
  \coordinate (draw-left-bottom) at
                ($(tag-left-bottom) + (\scaletag*1.5,\scaletag*1.5)$);
  \coordinate (white-left-bottom) at
                ($(tag-left-bottom) + (\scaletag*3,\scaletag*3)$);
  \foreach \i in {1,...,3} {
    \draw  (h\i) circle (2.5mm);
    \draw  ([xshift=-0.25cm]h\i) -- ([xshift=0.25cm]h\i);
    \draw  ([yshift=-0.25cm]h\i) -- ([yshift=0.25cm]h\i);
  }
  \fill [black] (tag-left-bottom) rectangle +(\scaletag*13.5,\scaletag*13.5);
  \fill [white] (white-left-bottom) rectangle +(\scaletag*7.5,\scaletag*7.5);
    \foreach \x/\y in @MARKER@ {
      %transform old tag creation to be rotated
      \pgfmathtruncatemacro{\xi}{6-\y};
      \pgfmathtruncatemacro{\yi}{\x};
      \fill [draw=black]
              ($(draw-left-bottom) + (1.5*\xi*\scaletag,1.5*\yi*\scaletag)$)
              rectangle +(\scaletag*1.5,\scaletag*1.5);}
\end{tikzpicture}
\end{document}
